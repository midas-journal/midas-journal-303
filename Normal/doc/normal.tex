%
% Complete documentation on the extended LaTeX markup used for Insight
% documentation is available in ``Documenting Insight'', which is part
% of the standard documentation for Insight.  It may be found online
% at:
%
%     http://www.itk.org/

\documentclass{InsightArticle}

\usepackage[dvips]{graphicx}

%%%%%%%%%%%%%%%%%%%%%%%%%%%%%%%%%%%%%%%%%%%%%%%%%%%%%%%%%%%%%%%%%%
%
%  hyperref should be the last package to be loaded.
%
%%%%%%%%%%%%%%%%%%%%%%%%%%%%%%%%%%%%%%%%%%%%%%%%%%%%%%%%%%%%%%%%%%
\usepackage[dvips,
bookmarks,
bookmarksopen,
backref,
colorlinks,linkcolor={blue},citecolor={blue},urlcolor={blue},
]{hyperref}

\usepackage{amsthm}
\theoremstyle{plain}
\newtheorem*{note}{Note}

%  This is a template for Papers to the Insight Journal.
%  It is comparable to a technical report format.

% The title should be descriptive enough for people to be able to find
% the relevant document.
\title{Surface Mesh Normals Filter}

% Increment the release number whenever significant changes are made.
% The author and/or editor can define 'significant' however they like.
\release{0.01}

% At minimum, give your name and an email address.  You can include a
% snail-mail address if you like.
\author{Arnaud Gelas, Alexandre Gouaillard and Sean Megason}
\authoraddress{Megason Lab, Harvard Medical School}



\begin{document}


\ifpdf
\else
   %
   % Commands for including Graphics when using latex
   %
   \DeclareGraphicsExtensions{.eps,.jpg,.gif,.tiff,.bmp,.png}
   \DeclareGraphicsRule{.jpg}{eps}{.jpg.bb}{`convert #1 eps:-}
   \DeclareGraphicsRule{.gif}{eps}{.gif.bb}{`convert #1 eps:-}
   \DeclareGraphicsRule{.tiff}{eps}{.tiff.bb}{`convert #1 eps:-}
   \DeclareGraphicsRule{.bmp}{eps}{.bmp.bb}{`convert #1 eps:-}
   \DeclareGraphicsRule{.png}{eps}{.png.bb}{`convert #1 eps:-}
\fi


\maketitle


\ifhtml
\chapter*{Front Matter\label{front}}
\fi


% The abstract should be a paragraph or two long, and describe the
% scope of the document.
\begin{abstract}
\noindent
We have previously developed a new surface mesh data structure in itk (\code{itk::QuadEdgeMesh}~\cite{itkQE}). In this document we describe a new filter (\code{itk::QENormalFilter}) to estimate normals for a given triangular surface mesh in this data structure. Here we describe the implementation and use of this filter for calculating normals of a \code{itk::QuadEdgeMesh}.
\end{abstract}

\tableofcontents

\section{Description}
This filter takes as input one triangular surface mesh (\code{itk::QuadEdgeMesh}~\cite{itkQE}) and returns one triangular surface mesh with face normals stored in the \code{CellDataContainer} and vertex normals stored in the \code{PointDataContainer}. It first computes the normal to all faces, and then calculates the normal for each vertex as the weighted sum of the normals of the neighboring faces~\cite{Jin05}.

\section{Implementation}
For a given triangular face on a given triangular surface oriented mesh, the normal is computed as the cross product of oriented vectors in order to make the orientation consistent. Then the vertex normal is computed as a weighted sum of the normal to the neighboring faces:
\begin{equation}
  \mathbf{n}_{\mathbf{v}} = \frac{\sum\limits_{i=0}^{\#f} w_i \cdot \mathbf{n}_i}{\left\|\sum\limits_{k=0}^{\#f} w_k \cdot \mathbf{n}_k \right\|}
\end{equation}
where $\#f$ is the number of faces around one given vertex $\mathbf{v}$, $w_i$ is a weight parameter which depends on the variable \code{m\_Weight}:
\begin{description}
  \item[GOURAUD: ] $w_i=1$ for any triangle~\cite{Gouraud71},
  \item[THURMER: ] $w_i$ is the angle of the considered triangle at the given vertex~\cite{Thurmer98},
  \item[AREA: ] $w_i$ is the area of the considered triangle.
 \end{description}

\begin{note}
  One can easily makes his or her own weight, or reimplements other weights (see~\cite{Jin05}), by modifying the method \code{itk::QENormalFilter::Weight}.
\end{note}

\section{Usage}
This filter is really easy to use, see given example \code{NormalFilter.cxx}.

\section*{Acknowledgment}
This work was funded by a grant from the NHGRI (P50HG004071-02) to found the Center for in toto genomic analysis of vertebrate development.

\bibliographystyle{plain}
\bibliography{normal}


\end{document}

